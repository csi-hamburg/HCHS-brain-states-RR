\renewcommand\cellset{\renewcommand\arraystretch{0.5}%
    \setlength\extrarowheight{0pt}}
\begin{table}[bt]
    \scriptsize
    \begin{fullwidth}
        \begin{threeparttable}
            \begin{spacing}{0.8}\centering
                \begin{tabularx}{1.3\textwidth}{p{.2\linewidth} p{.1\linewidth} p{.09\linewidth} p{.15\linewidth} p{.09\linewidth} p{.14\linewidth} p{.08\linewidth}}
                    \toprule
                    Question   & Hypothesis     & Sampling plan     & Analysis plan   & Rationale for deciding the sensitivity of the test & Interpretation given different outcomes & Theory that could be shown wrong by the outcome \\
                    \midrule
                    Is severity of cerebral small disease, quantified by the volume supratentorial white matter hyperintensities of presumed vascular origin (WMH), associated with time spent in high-occupancy brain states, defined by resting-state functional MRI & 
                    Higher WMH volume is associated with lower average occupancy of the two highest-occupancy brain states & 
                    Available subjects with clinical and imaging data from the HCHS \citep{Jagodzinski2020-lx} & 
                    Standardized preprocessing of structural and functional MRI data • automatic quantification of WMH • co-activation pattern analysis • multivariable generalised regression analyses &
                    Tradition &
                    \makecell[tp{\linewidth}]{$P<0.05$ --> rejection of the null hypothesis of no association between cSVD and fractional occupancy;\\ $P>0.05$ --> insufficient evidence to reject the null hypothesis}    &
                    Functional brain dynamics are not related to subcortical ischemic vascular disease.    \\
                    \bottomrule
                \end{tabularx}
            \end{spacing}
            \bigskip
            \caption{Study Design Template}
        \end{threeparttable}
    \end{fullwidth}
\end{table}