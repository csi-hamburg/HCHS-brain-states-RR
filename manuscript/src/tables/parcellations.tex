\renewcommand\cellset{\renewcommand\arraystretch{0.5}%
    \setlength\extrarowheight{0pt}}
\begin{table}[bt]
    \begin{threeparttable}
        \begin{subtable}[t]{.75\textwidth}
            \label{tab:parcellations2}
            % Use "S" column identifier to align on decimal point 
            \begin{tabularx}{\textwidth}{l l l}
                \toprule
                \textbf{Name of the atlas}  & \textbf{\#parcels}                                  & \textbf{Reference}          \\
                \midrule
                Desikan--Killiany           & \num{86}                                            & \cite{desikan2006automated} \\
                AAL                         & \num{116}                                           & \cite{tzourio2002automated} \\
                Harvard--Oxford             & \num{112}                                           & \cite{makris2006decreased}  \\
                glasser360                  & \num{360}                                           & \cite{Glasser2016-ia}       \\
                gordon333                   & \num{333}                                           & \cite{Gordon2016-wy}        \\
                power264                    & \num{264}                                           & \cite{Power2011-xf}         \\
                schaefer\{N\}               & \makecell[lt]{\num{100} \\ \num{200}\\ \num{400}}   & \cite{Schaefer2018-bo}      \\
                \bottomrule
            \end{tabularx}
            \begin{tablenotes}
                \item{AAL:} Automatic Anatomical Labelling
            \end{tablenotes}
            \caption{Parcellations}
        \end{subtable}
        \qquad
        \begin{subtable}[t]{.75\textwidth}
            \label{tab:parcellations}
            % Use "S" column identifier to align on decimal point 
            \begin{tabularx}{\textwidth}{l l l}
                \toprule
                \textbf{Design}        & \textbf{Reference}                 \\
                \midrule
                24p                    & \cite{friston1996movement}         \\
                24p + GSR              & \cite{macey2004method}             \\
                36p              & \cite{satterthwaite2013improved}   \\
                36p + spike regression & \cite{cox1996afni}                 \\
                36p + despiking        & \cite{satterthwaite2013improved}   \\
                36p + scrubbing  & \cite{power2014methods}            \\
                aCompCor               & \cite{muschelli2014reduction}      \\
                tCompCor               & \cite{behzadi2007component}        \\
                AROMA                  & \cite{pruim2015ica}                \\
                \bottomrule
            \end{tabularx}
            \begin{tablenotes}
                \item GSR: Global signal regression, AROMA: Automatic Removal of Motion Artifacts
            \end{tablenotes}
            \caption{Confound regression strategies, adapted from \citep{Ciric2017-cl}}
        \end{subtable}
        \makeatletter\def\TPT@hsize{}\makeatletter
    \end{threeparttable}
    \mycaption{Multiverse analysis}{Overview over different brain parcellations and confound regression strategies implemented using xcpEngine \citep{ciric2018mitigating}. A total of $9\times 9=81$ analytical combinations were explored to assess the robustness of our results with respect to these processing choices.}
    \label{tab:multiverse}
\end{table}