\section{Introduction} \label{intro}
Cerebral small vessel disease (cSVD) is an arteriolopathy of the brain, associated with age and common cardiovascular risk factors \citep{Wardlaw2013-yd}.
cSVD predisposes to ischemic, in particular lacunar, stroke and may lead to cognitive impairment and dementia \citep{Cannistraro2019-ly}.
Neuroimaging findings in cSVD reflect its underlying pathology \citep{Wardlaw2015-ri} and include white matter hyperintensities (WMH) and lacunes of presumed vascular origin, small subcortical infarcts and microbleeds, enlarged perivascular spaces as well as brain atrophy \citep{Wardlaw2013-sc}.
However, the extent of visible cSVD features on magnetic resonance imaging (MRI) is an imperfect predictor of the severity of clinical sequelae \citep{Das2019-pc}, and our understanding of the causal mechanisms linking cSVD-associated brain damage to clinical deficits remains limited \citep{Bos2018-qj}.

Recent efforts have concentrated on exploiting network aspects of the structural \citep{Tuladhar2016-ae,Tuladhar2020-fp,Lawrence2018-ti} and functional \citep{Dey2016-qg,Schulz2021-ho} organization of the brain to understand the relation between cSVD and clinical deficits in cognition and other domains reliant on distributed processing.
Reduced structural network efficiency has repeatedly been described as a causal factor in the development of cognitive impairment, in particular executive dysfunction, in cSVD \citep{Lawrence2014-xp,Shen2020-yv,Reijmer2016-wm,Prins2005-ej}.
Findings with respect to functional connectivity results, on the other hand, are more heterogeneous, perhaps due to its limited reproducibility in the presence of cSVD and dependence on arbitrary processing choices \citep{Lawrence2018-sv,Gesierich2020-db}.
As a promising new avenue, time-varying, or dynamic, functional connectivity approaches have more recently been explored in patients with subcortical ischemic vascular disease \citep{Yin2022-cv,Xu2021-ib}.

In the present paper, we aim to replicate and extend the main results of \citep{Schlemm2022-he};
in this recent study, the authors analyzed MR imaging and clinical data from the prospective Hamburg City Health Study (HCHS, \citep{Jagodzinski2020-lx}) using a coactivation pattern approach to define discrete brain states and found associations between the WMH load, time spent in high-occupancy brain states characterized by activation or suppression of the default mode network (DMN) and executive dysfunction.

Our primary hypothesis is that the volume of supratentorial white matter hyperintensities is associated with the fractional occupancy (defined below) of DMN-related brain states in a middle-aged to elderly population mildly affected by cSVD.
Our second hypothesis is that this fractional occupancy is associated with executive dysfunction, measured as the time to complete part B of the trail making test (TMT).

Both hypotheses will be tested in an independent subsample of the HCHS study population using the same imaging protocols, examination procedures and analysis pipelines as in \citep{Schlemm2022-he}.
The robustness of associations will be explored in a multiverse approach by varying key steps in the analysis pipeline.
