\begin{abstract}

{\bf Objective:} To replicate recent findings on the association between the extent of cerebral small vessel disease (cSVD), functional brain network dedifferentiation, and cognitive impairment.

{\bf Methods:} We analyzed demographic, imaging, and behavioral data from the prospective population-based Hamburg City Health Study.
Using a fully prespecified analysis pipeline, we estimated discrete brain states from structural and resting-state functional magnetic resonance imaging (MRI).
In a multiverse analysis, we varied brain parcellations and functional MRI confound regression strategies.
The severity of cSVD was operationalized as the volume of white matter hyperintensities of presumed vascular origin.
Processing speed and executive dysfunction were quantified using the Trail Making Test (TMT).

{\bf Hypotheses:} We hypothesized a) that a greater volume of supratentorial white matter hyperintensities would be associated with less time spent in functional MRI-derived brain states of high fractional occupancy;
and b) that less time spent in these high-occupancy brain states is associated with a longer time to completion in part B of the TMT. 

{\bf Results:} High-occupancy brain states were characterized by activation or suppression of the default mode network. 
Every \num{5.1}-fold increase in WMH volume was associated with a \num{0.94}-fold reduction in the odds of occupying DMN-related brain states (P \num{5.01e-8}). Every \qty{5}{\percent} increase in time spent in high-occupancy brain states was associated with a \num{0.98}-fold reduction in the TMT-B completion time (P \num{0.0116}). 
Findings were robust across most brain parcellations and confound regression strategies.

{\bf Conclusion:} We successfully replicated previous findings on the association between cSVD, functional brain occupancy, and cognition in an independent sample. 
The data provide further evidence for a functional network dedifferentiation hypothesis of cSVD-related cognitive impairment. 
Further research is required to elucidate the mechanisms underlying these associations. 
\end{abstract}

