\begin{abstract}

{\bf Objective:} To replicate recent findings about the association between the extent of cerebral small vessel disease (cSVD), functional brain network dedifferentiation and cognitive impairment.

{\bf Methods:} We will analyze demographic, imaging and behavioral data from the prospective population-based Hamburg City Health Study.
Using a fully prespecified analysis pipeline, we will estimate discrete brain states from structural and resting-state functional magnetic resonance imaging (MRI).
In a multiverse analysis we will vary brain parcellations and functional MRI confound regression strategies.
Severity of cSVD will be operationalised as the volume of white matter hyperintensities of presumed vascular origin.
Processing speed and executive dysfunction are quantified by the trail making test (TMT).

{\bf Hypotheses:} We hypothesize a) that greater volume of supratentorial white matter hyperintensities is associated with less time spent in functional MRI-derived brain states of high fractional occupancy;
and b) that less time spent in these high-occupancy brain states is associated with longer time to completion in part B of the TMT. 
\end{abstract}