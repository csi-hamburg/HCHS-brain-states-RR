\section{Summary and Discussion} \label{discussion}

In this pre-registered cross-sectional study we replicated the key findings of \cite{Schlemm2022-he} in an independent population-based sample of \num{1651} middle-aged to elderly participants of the Hamburg City Health Study.

First, we confirmed that the severity of cerebral small vessel disease is associated with the time spent in high-occupancy brain states, defined by functional MRI.
More precisely, we showed that every 5.1-fold increase in the volume of supratentorial white matter hyperintensities of presumed vascular origin (WMH) was associated with a 0.95-fold reduction in the odds of occupying a brain state characterized by activation or suppression of the default-mode network, at any given time during the resting-state scan.

Second, we confirmed that the time spent in high-occupancy brain states at rest is associated with cognitive performance. 
More precisely, a 5\%-reduction in the fractional occupancy of DMN-related brain states was associated with a \num{1.02}-fold increase in the time to complete part B of the trail making test (TMT).

In a pre-planned multiverse analysis, findings relating to our primary and, to a lesser extent, secondary hypotheses were robust with respect to variations in brain parcellations and confound regression strategies. Inconsistent results were found with the Desikan--Killiany parcellation, likely reflecting the notion that the spatial resolution and functional specificity of this coarse, structurally defined atlas are inadequate for analyzing functionally defined brain states.  Across brain parcellations, effect sizes  were smaller with the ICA-AROMA confound regression strategy and failed to reach nominal statistical significance. This might be due to a relatively large residual motion component in measures of dynamical functional Connectivity after de-noising with ICA-AROMA, as described previously \citep{lydon2019evaluation}.

We also confirmed across several brain parcellation resolutions that high-occupancy states at rest are characterized by either activation or suppression of the default mode network, reflecting its role as the predominant task-negative brain network.

In unplanned, exploratory analyses, we described the association between brain state dynamics and cognitive measures other than executive function and processing speed and reported a strong, preliminary association between time spent in high-occupancy states and delayed word recall.

We further explored, and report in the Supplementary appendix, the effect of motion; results relating to our primary and, to a lesser extent, secondary, hypotheses were robust to additional, unplanned adjustments for DVARS, RMSD, and mean framewise displacement.

The presented results provide robust evidence for a behaviorally relevant association between cerebral small vessel disease and functional brain network dedifferentiation. 

Further research is required to replicate our findings in different populations, such as those affected more severely by cSVD or cognitive impairment, or being studied using different imaging protocols, to determine the generalizability of our findings with respect to varying operationalizations of the notions of cSVD, brain state, and cognition, and to understand the mechanisms underlying the reported associations.

